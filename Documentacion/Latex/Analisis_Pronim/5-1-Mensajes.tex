%!TEX root = ejemplo.tex
%=========================================================
\section{Catálogo de mensajes}	
\label{sec:mensajes}

	En esta sección se describen todos los mensajes que aparecen en el sistema. Para cada mensaje se especifica:
	 
	\begin{description}\itemsep0em
		\item[Id:] Identificador del mensaje de la forma ``MSG XX'' y descripción corta del mismo.
		\item[Tipo:] Tipo del mensaje el cual puede ser: 
		\begin{description}
			\item[Normal:] Mensaje que informa al usuario una instrucción o el estado interno que guarda el sistema, suele tener un color {\color{msgNormalColor}Azul}.
			\item[Éxito:] Mensaje que informa al usuario sobre una acción realizada, sirve para confirmar el correcto funcionamiento del sistema. Se presentan con un color {\color{msgInfoColor}Verde}.
			\item[Atención:] Mensaje que tiene como finalidad llamar la atención del usuario a una situación que requiere su intervención, por ejemplo cuando una actividad ha generado un efecto colateral o se realizará una acción destructiva y no reversible. Se presentan con un color {\color{msgWarningColor}Naranja}.
			\item[Error:] Mensaje que informa al usuario un fallo en en una operación o un impedimento para realizarla, por ejemplo: cuando no se puede efectuar la acción solicitada, cuando un dato falta o tiene un formato no aceptado por el sistema. Se presentan con un color {\color{msgErrorColor}Rojo}.
		\end{description}
		\item[Propósito:] Explicación del propósito del mensaje.
		\item[Redacción:] Redacción del mensaje.
		\item[Parámetros:] En caso de que el mensaje pueda variar se especifican los casos y la forma en que debe adaptarse la redacción
		\item[Ejemplos:] Ejemplos de como debe renderizarse el mensaje.
	\end{description}


\subsection{Lista de mensajes}

%msgNormalColor
%msgInfoColor
%msgWarningColor
%msgErrorColor

%Mensajes de error relacionado al inicio de secion (Origen CU-001)

\begin{cdtMessage}[msgInfoColor]{MSG-001}{Bienvenida al usuario}
	\item[Propósito:] Indicar al usuario que ha ingresado satisfactoriamente al sistema.
	\item[Redacción:] Bienvenido $<$nombre$>$ $<$Primer Apellido$>$.
	\item[Parámetros:] \hspace{1cm}
	\begin{itemize}
		\item $<$nombre$>$ \hyperlink{Usuario.nombre}{Nombre} del Usuario.
  \item $<$Primer Apellido$>$ \hyperlink{Usuario.primerApellido}{Primer Apellido} del Usuario.
	\end{itemize}
	\item[Ejemplos:] Bienvenido Juan Pérez.
\end{cdtMessage}

\begin{cdtMessage}[msgErrorColor]{MSG-002}{Usuario no registrado} 
	\item[Propósito:] Indica que el usuario ingresado no existe en el sistema.
	\item[Redacción:] El usuario $<$Identificador$>$ no se encuentra registrado.
	\item[Parámetros:] \hspace{1cm}
	\begin{itemize}
		\item $<$Identificador$>$ \hyperlink{Usuario.ID}{Identificador} del Usuario.
	\end{itemize}
	\item[Ejemplos:] El usuario 123123 no se encuentra registrado.
\end{cdtMessage}

\begin{cdtMessage}[msgErrorColor]{MSG-003}{Cuenta inactiva} 
	\item[Propósito:] Indicar al usuario que la cuenta especificada se encuentra inactiva.
	\item[Redacción:] La cuenta especificada $<$ID$>$ se encuentra inactiva.
	\item[Parámetros:] \hspace{1cm}
	\begin{itemize}
		\item $<$ID$>$ \hyperlink{Usuario.ID}{Identificador} del Usuario.
	\end{itemize}
	\item[Ejemplos:] La cuenta especificada 123123 se encuentra inactiva.
\end{cdtMessage}

\begin{cdtMessage}[msgErrorColor]{MSG-004}{Error de inicio de sesión}
	\item[Propósito:] Indica al usuario que la contraseña introducida es incorrecta.
	\item[Redacción:] La contraseña ingresada es incorrecta.
	\item[Parámetros:] No aplica.
	\item[Ejemplos:] La contraseña ingresada es incorrecta.
\end{cdtMessage}

\begin{cdtMessage}[msgErrorColor]{MSG-005}{Faltan Campos Obligatorios} 
	\item[Propósito:] Indicar al usuario que olvido agregar algún campo obligtorio.
	\item[Redacción:] El campo $<$campo$>$ es obligatorio y se encuentra vacio.
	\item[Parámetros:] \hspace{1cm}
	\begin{itemize}
		\item $<$campo$>$ de la IU.
	\end{itemize}
	\item[Ejemplos:] El campo contraseña es obligatorio y se encuentra vacio.
\end{cdtMessage}

%Mensajes de informacion no compatible con los filtros (CU002)

\begin{cdtMessage}[msgInfoColor]{MSG-006}{No se encontró información} 
	\item[Propósito:] Indicar al usuario que no existen elementos que cumplan con los filtros establecidos.
	\item[Redacción:] Ningún elemento cumple con los filtros puestos.
	\item[Parámetros:] \hspace{1cm}
	\begin{itemize}
		\item No aplica.
	\end{itemize}
	\item[Ejemplos:] Se puso de filtro un nombre de escuela que no existe, ergo no muestra ningún elemento.
\end{cdtMessage}

%Mensajes de error relacionados con tiempo (Duda de que CU de origen)
\begin{cdtMessage}[msgErrorColor]{MSG-007}{Campo con formato incorrecto} 
	\item[Propósito:] Indicar al usuario que el contenido de uno de sus campos no cumple con el formato requerido.
	\item[Redacción:] El campo $<$campo$>$ no cumple con el formato especificado, favor de corroborarlo.
	\item[Parámetros:] \hspace{1cm}
	\begin{itemize}
		\item $<$campo$>$ de la IU.
	\end{itemize}
	\item[Ejemplos:] El campo CURP no cumple con el formato especificado, favor de corroborarlo.
\end{cdtMessage}
\begin{cdtMessage}{MSG-008}{Tiempo restante para terminar un proceso} 
	\item[Propósito:] Indicar al usuario el tiempo restante para terminar una operación limitada en el tiempo como la reinscripción.
	\item[Redacción:] Quedan $<$tiempo$>$ para terminar la $<$operación$>$.
	\item[Parámetros:] \hspace{1cm}
	\begin{itemize}
		\item $<$tiempo$>$ Tiempo faltante para la operació especificando días, horas minutos y segundos.
		\item 
	\end{itemize}
	\item[Ejemplos:] \hspace{1cm}
	\begin{itemize}
		\item Quedan 45 días, 2 horas 12 minutos y 45 segundos para iniciar tu reinscripción.
		\item Quedan 2 minutos y 32 segundos para terminar tu reinscripción
	\end{itemize}
\end{cdtMessage}

\begin{cdtMessage}[msgInfoColor]{MSG-009}{Operación exitosa}
	\item[Propósito:] Informar al usuario que la operación solicitada ha sido ejecutada con éxito.
	\item[Redacción:] $<$Artículo$>$ $<$Operación$>$ del $<$Entidad$>$ $<$Identificador$>$ se realizó con éxito.
	\item[Parámetros:]\hspace{1pt}
	\begin{itemize}
		\item $<$Artículo$>$ $<$Operación$>$ se refiere a la operación realizada.
		\item $<$Entidad$>$ $<$Identificador$>$ se refiere al elemento del negocio donde recayó la operación, indicando el tipo del objeto y un dato que el usuario pueda usar para identificarlo.
	\end{itemize}
	\item[Ejemplo:] Algunos ejemplos son
	\begin{itemize}
		\item El registro del Alumno 342343 se realizó con éxito.
		\item La eliminación de la Tarea ``documentar el proceso'' se ha realizado con éxito.
	\end{itemize}
\end{cdtMessage}
\begin{cdtMessage}[msgErrorColor]{MSG-010}{Clave del Centro de Trabajo no Prestada} 
	\item[Propósito:] Indicar al usuario que la clave de centro de trabajo proporcionada no permite ser prestado.
	\item[Redacción:] La CCT proporcionada no permite ser prestada, por favor ponerse en contacto con el director del Centro de Trabajo propietario.
	\item[Parámetros:] \hspace{1cm}
	\begin{itemize}
		\item No aplica.
	\end{itemize}
	\item[Ejemplos:] La CCT proporcionada no permite ser prestada, por favor ponerse en contacto con el director del Centro de Trabajo propietario.
\end{cdtMessage}
\begin{cdtMessage}[msgInfoColor]{MSG-011}{registro exitoso}
	\item[Propósito:] Indicar al usuario que ha registrado satisfactoriamente un nuevo registro.
	\item[Redacción:] Se ha completado el registro de $<$nombre$>$ .
	\item[Parámetros:] \hspace{1cm}
	\begin{itemize}
		\item $<$nombre$>$ \hyperlink{Usuario.nombre}{Nombre} del Usuario.
	\end{itemize}
	\item[Ejemplos:] Se ha completado el registro de un Centro de Trabajo
\end{cdtMessage}
\begin{cdtMessage}[msgErrorColor]{MSG-012}{Fecha anterior a la fecha actual} 
	\item[Propósito:] Indicar al usuario que la fecha seleccionada es anterior a la fecha actual.
	\item[Redacción:] La fecha proporcionada es anterior al día de hoy por lo que no se puede admitir.
	\item[Parámetros:] \hspace{1cm}
	\begin{itemize}
		\item No aplica
	\end{itemize}
	\item[Ejemplos:]  La fecha proporcionada es anterior al día de hoy por lo que no se puede admitir.
\end{cdtMessage}