% !TeX root = ../proyecto.tex
%--------------------------------------
\section{IU005 Interfaz (Registro Centro de Trabajo)}

\subsection{Objetivo}
	\cdtInstrucciones{Describa el objetivo, propósito o función de la pantalla.}
 Esta Interfaz tiene como objetivo permitir al usuario introducir nuevos centros de trabajo, claves de centro de trabajo y/o prestamos de las mismas al programa.

\subsection{Diseño}
	\cdtInstrucciones{Describa brevemente los elementos de la pantalla y como de be usarse a manera de manual de usuario.} Esta pantalla \IUref{IU005}{Registro Centro de Trabajo} aparece al seleccionar la opción en el menú, contiene 5 campos de texto que deben ser llenados con los datos del Centro de trabajo, tambien tiene 2 seletores y un boton para intentar guardar los datos.
\clearpage
\IUfig[.5]{IU005}{IU005}{Registro Centro de Trabajo.}

%\IUfig[ancho de la figura: valor entre 1 y .1]{Nombre corto de la pantalla sin espacions ni acentos}{IUXX}{Nombre largo de la pantalla.}

\subsection{Salidas}

	\cdtInstrucciones{Liste las salidas de la interfaz. Si coinciden con las del caso de uso solo indiquelo. Esta ,ista debe incluir los mensajes}

	\begin{itemize}
		\item Coincide con el caso.
	\end{itemize}
	
\subsection{Entradas}

	\cdtInstrucciones{Liste las entradas de la interfaz. Si coinciden con las del caso de uso solo indiquelo.}
	\begin{itemize}
		\item Coincide con el caso.
	\end{itemize}

\subsection{Comandos}
	\cdtInstrucciones{Describa cada control (botónes, areas de drag and drop, componentes interactivos, animaciones, etc.) que se puede utilizar dentro de la pantalla indicando o que hacen y si cambia de pantalla.}

\begin{itemize}
	\item \IUbutton{Registrar}: Verifica que los datos introducidos sean validos y tengan el formato correcto, de ser asi guardara el nuevo centro de trabajo limpiando el formulario.
	
\end{itemize}

