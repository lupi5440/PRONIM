% !TeX root = ../proyecto.tex
%--------------------------------------
\section{IUX Interfaz (nombre de la interfaz)}

\subsection{Objetivo}
	\cdtInstrucciones{Describa el objetivo, propósito o función de la pantalla.}

\subsection{Diseño}
	\cdtInstrucciones{Describa brevemente los elementos de la pantalla y como de be usarse a manera de manual de usuario.} Esta pantalla \IUref{IU23}{Pantalla de Control de Acceso} aparece al iniciar el sistema, para ingresar ... 

\IUfig[.7]{Login}{IU23}{Pantalla de Control de Acceso.}

%\IUfig[ancho de la figura: valor entre 1 y .1]{Nombre corto de la pantalla sin espacions ni acentos}{IUXX}{Nombre largo de la pantalla.}

\subsection{Salidas}

	\cdtInstrucciones{Liste las salidas de la interfaz. Si coinciden con las del caso de uso solo indiquelo. Esta ,ista debe incluir los mensajes}

	\begin{itemize}
		\item Descripción de salida.
	\end{itemize}
	
\subsection{Entradas}

	\cdtInstrucciones{Liste las entradas de la interfaz. Si coinciden con las del caso de uso solo indiquelo.}
	\begin{itemize}
		\item Descripción de salida.
	\end{itemize}

\subsection{Comandos}
	\cdtInstrucciones{Describa cada control (botónes, areas de drag and drop, componentes interactivos, animaciones, etc.) que se puede utilizar dentro de la pantalla indicando o que hacen y si cambia de pantalla.}

\begin{itemize}
	\item \IUbutton{Entrar}: Verifica que el Estudiante se encuentre registrado y la contraseña sea la correcta. Si la verificación es correcta, se muestra la \IUref{UI32}{Pantalla de Selección de Seminario}.
	\item \IUbutton{Ayuda}: Muestra la ayuda de esta pantalla \IUref{IU50}{Pantalla de Ayuda}.
\end{itemize}

