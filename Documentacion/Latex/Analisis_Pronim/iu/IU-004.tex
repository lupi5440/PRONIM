% !TeX root = ../proyecto.tex
%--------------------------------------
\section{IU004 Interfaz (Recuperar contraseña)}

\subsection{Objetivo}
	\cdtInstrucciones{Describa el objetivo, propósito o función de la pantalla.}
Esta Interfaz tiene como objetivo permitir al usuario recuperar su contraseña si la olvido pero recuerda su identificador y su correo.
\subsection{Diseño}
	\cdtInstrucciones{Describa brevemente los elementos de la pantalla y como de be usarse a manera de manual de usuario.} Esta pantalla \IUref{IU004}{Recuperar contraseña} contiene solo dos campos de texto y un botón para enviar el formulario .

\IUfig[.7]{IU004}{IU004}{Recuperar contraseña.}

%\IUfig[ancho de la figura: valor entre 1 y .1]{Nombre corto de la pantalla sin espacions ni acentos}{IUXX}{Nombre largo de la pantalla.}

\subsection{Salidas}

	\cdtInstrucciones{Liste las salidas de la interfaz. Si coinciden con las del caso de uso solo indiquelo. Esta ,ista debe incluir los mensajes}

	\begin{itemize}
		\item Igual al caso de uso
	\end{itemize}
	
\subsection{Entradas}

	\cdtInstrucciones{Liste las entradas de la interfaz. Si coinciden con las del caso de uso solo indiquelo.}
	\begin{itemize}
		\item Igual al caso de uso.
	\end{itemize}

\subsection{Comandos}
	\cdtInstrucciones{Describa cada control (botónes, areas de drag and drop, componentes interactivos, animaciones, etc.) que se puede utilizar dentro de la pantalla indicando o que hacen y si cambia de pantalla.}

\begin{itemize}
	\item \IUbutton{Recuperar}: Verifica los datos del formulario esten llenos, el id exista, la cuenta no este suspendida, y que el correo coincida con el guarddo. Si la verificación es correcta, se muestra la \IUref{IU-001}{Control de acceso}.
	
\end{itemize}

