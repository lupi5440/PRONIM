% !TeX root = ../proyecto.tex
%--------------------------------------
\section{IU003 Interfaz (Pagina Principal)}

\subsection{Objetivo}
	\cdtInstrucciones{Describa el objetivo, propósito o función de la pantalla.}
Esta Interfaz tiene como objetivo permitir al usuario una facil navegación a traves del sistema y las funcionalidades que el puede usar.
\subsection{Diseño}
	\cdtInstrucciones{Describa brevemente los elementos de la pantalla y como de be usarse a manera de manual de usuario.} Esta pantalla \IUref{IU003}{Pantalla Principal} aparece al acceder al sistema, consta de un menu desplegable con las siguientes categorias, nombre, materias,escuela .

\IUfig[.7]{IU003}{IU003}{Pantalla Principal.}

%\IUfig[ancho de la figura: valor entre 1 y .1]{Nombre corto de la pantalla sin espacions ni acentos}{IUXX}{Nombre largo de la pantalla.}

\subsection{Salidas}

	\cdtInstrucciones{Liste las salidas de la interfaz. Si coinciden con las del caso de uso solo indiquelo. Esta ,ista debe incluir los mensajes}

	\begin{itemize}
		\item Notificaciones
	\end{itemize}
	
\subsection{Entradas}

	\cdtInstrucciones{Liste las entradas de la interfaz. Si coinciden con las del caso de uso solo indiquelo.}
	\begin{itemize}
		\item Ninguna.
	\end{itemize}

\subsection{Comandos}
	\cdtInstrucciones{Describa cada control (botónes, areas de drag and drop, componentes interactivos, animaciones, etc.) que se puede utilizar dentro de la pantalla indicando o que hacen y si cambia de pantalla.}

\begin{itemize}
	\item \IUbutton{Entrar}: Verifica los datos del formulario esten llenos, el id exista, la contraseña coincida con la guardada y la cuenta no este suspendida, ademas busca el rol del perfil. Si la verificación es correcta, se muestra la \IUref{IU-003}{Pantalla Principal}.
	\item \IUbutton{Recuperr Contraseña}:Se muestra la \IUref{IU-004}{Recuperar contraseña}.
\end{itemize}

