%!TEX root = ../proyecto.tex

% Plantilla para caso de uso sencillo con ejemplos de comandos e intrucciones.
%-------------------------------------- COMIENZA descripción del caso de uso.

%\begin{UseCase}[archivo de imágen]{UCX}{Nombre del Caso de uso}{
%--------------------------------------
\begin{UseCase}{CU-006}{Registrar Centro de Trabajo}{
Cuando un Centro de Trabajo sea añadida al PRONIM el Coordinador Estatal proveera al sistema los datos del Centro de Trabajo y el sistema los almacenará para su posterior consulta y uso.
	% La descripción debe describir el evento de inicio del caso de uso, Breve descripción de la trayectoria y estado final del caso de uso.
}
	\UCitem{Versión}{\color{Gray}
		0.1	% Ponga un número de versión, 
	}\UCitem{Autor}{\color{Gray}
		Agustin Flores Lopez. % Analista responsable de especificar el CU
	}\UCitem{Supervisa}{\color{Gray}
		Nombre del analista revisor. % Analista responsable de verificar que está correcto.
	% TODO: Dar de alta al actor Usuario
	}\UCitem{Actor}{
		\hyperlink{CoordinadorEstatal}{Coordinador Estatal} % No olvide dar de alta el actor.
	}\UCitem{Propósito}{\begin{Titemize}%Indique los fines, objetivos, propósitos o valores agregados del Caso de uso.
		\Titem Mantener actualizado el catalogo de Centros de Trabajo asociados al PRONIM.
		\Titem Permitir llevar el seguimiento de los alumnos que se registren en los nuevos centros de trabajo.
	\end{Titemize}
	}\UCitem{Entradas}{\begin{Titemize}
		% TODO: Dar de alta las entidades que se listan.
		\Titem \hyperlink{CT.Calle}{Calle} donde se ubica. % El identificador no acepta acentos, espacios ni eñes.
		\Titem \hyperlink{CT.Numero}{Numero} donde se ubica.
        \Titem \hyperlink{CT.Colonia}{Colonia} donde se ubica.
        \Titem \hyperlink{CT.Municipio}{Municipio} donde se ubica.
        \Titem \hyperlink{CT.Nombre}{Nombre} del centro de trabajo.
        \Titem \hyperlink{Estado.Nombre}{Estado} donde se ubica.
        \Titem\hyperlink{CCT.CCT}{Clave} del centro de trabajo.
        \Titem \hyperlink{Formacion.Nombre}{Formación} que imparte.
        % Liste todos los datos de entrada
		\end{Titemize}
	}\UCitem{Origen}{\begin{Titemize}
		\Titem Se introducen desde el teclado. % Indique por que medio se introducen los datos, 
		\Titem Se seleccionan de listas desplegables.          % Si es ḿas de uno indique que datos corresponden en cada medio de entrada.
	\end{Titemize}
	}\UCitem{Salidas}{\begin{Titemize}
		% TODO: Dar de alta las entidades que se listan.
		\Titem \hyperlink{CT.Calle}{Calle} donde se ubica. % El identificador no acepta acentos, espacios ni eñes.
		\Titem \hyperlink{CT.Numero}{Numero} donde se ubica.
        \Titem \hyperlink{CT.Colonia}{Colonia} donde se ubica.
        \Titem \hyperlink{CT.Municipio}{Municipio} donde se ubica.
        \Titem \hyperlink{Estado.Nombre}{Estado} donde se ubica.
        \Titem \hyperlink{Formacion.Nombre}{Formación} que imparte.
        \Titem\hyperlink{CCT.CCT}{Clave} del centro de trabajo.
		\Titem Mensajes de error. % Indique por que medio se introducen los datos, 
		\Titem Datos que aparecen en pantalla.          % Si es ḿas de uno indique que datos corresponden en cada medio de entrada.
	\end{Titemize}
	}\UCitem{Destino}{\begin{Titemize}
		\Titem Se muestra en la pantalla \IUref{IU005}{Registro Centro de Trabajo}.. % Indique por que medio se muestran los datos,         % Si es ḿas de uno indique que datos corresponden en cada medio de entrada.
	\end{Titemize}
	}\UCitem{Precondiciones}{\begin{Titemize}
		% Incluya Precondiciones lógicas, de negocio e incluso las que debe atender el usuario. 
		% Muchas precondiciones provienen de reglas de negocios, otras estarán asociadas a manejo de errores
		% Otras están relacionadas con casos de uso que deben ejecutarse previamente, como registrar un producto.
		\Titem El usuario debe haber iniciado sesión.
  \Titem Debe haber al menos un estado registrado.
  \Titem Debe haber al menos una formación registrado.
  \Titem No debe haber sido registrado con anterioridad.
  
	\end{Titemize}
	}\UCitem{Postcondiciones}{\begin{Titemize}
		% Indique todas las postcondiciones
		% por ejemplo, Cambios en el sistema
		% Cambios en la BD una vez terminado el CU
		% Efectos colaterales
		% Condiciones de término.
		\Titem Se guardara un nuevo centro de trabajo.
        \Titem Se guardara una nueva CCT o un nuevo prestamo.
        \Titem Se creara un puesto de director vacio.
        \Titem Se permitira nuevos registros de profesores y alumnos para ese Centro de Trabajo.
	\end{Titemize}
	}\UCitem{Errores}{\begin{Titemize}
		% Escriba todos los errores que puedan ocurrir en el sistema, para cada error recuerde:
		% Punerle un identificador
		% Describir la condición o escenario que detona el error
		% Describa la forma en que debe reaccionar el sitema: si la reaccion corresponde a varios pasos use mejor una trayectoria alternativa.
		% Relacione el error con la trayectoria principal.
		\Titem {\bf \hypertarget{CU004.E1}{E1}}: Si el usuario olvida ingresar alguno de los campos obligatorios, el sistema mostrara el \MSGref{MSG-005}{Faltan rellenar campos obligatorios}. y regresa al paso \ref{UC4.Datos}.
		\Titem {\bf \hypertarget{CU004.E2}{E2}}: Si la clave de centro de trabajo no cumple con el formato, el sistema mostrara el \MSGref{MSG-007}{Campo con formato incorrecto}  y regresa al paso \ref{UC4.Datos}.
  \Titem {\bf \hypertarget{CU004.E3}{E3}}: Si la clave de centro de trabajo esta repetida y no tiene el estado de prestamo activado, el sistema mostrara el \MSGref{MSG-010}{Clave del Centro de Trabajo no Prestada} y regresa al paso \ref{UC4.Datos}.
  
	\end{Titemize}
	}\UCitem{Tipo}{
		% Especifique el tipo de caso de us, puede ser: "Caso de uso primario" o 
		% "Viene de \\hyperref{CUY}{CUY nombre del CU}" cuando se desprende desde otro caso de uso mediante un extends.
		Caso de uso primario
	}\UCitem{Observaciones}{
		% Indique las observaciones al caso de uso, las cuales pueden ser:
		% - Ninguna
		% - Dudas sobre el procedimiento o la especificación.
		% - Issues detectados
		% - Suposiciones realizadas.
		% - Cualquier otra especificacion que considere pertinente que no pudo colocarse en los demás atributos del Caso de uso
		% - Aclaraciones.
		% - Notas para el usuario o desarrollador.
		% - Pendientes (TODO's) en caso de no usar los comentarios.
	}
\end{UseCase}

%--------------------------------------
\begin{UCtrayectoria}
	% Cada paso debe inicair con un Verbo en infinitivo, siempre especificando el objetivo del paso mas la accion en concreto.
	% \UCpaso[\UCactor] se refiere al actor y \UCpaso se refiere al sistema.
	% A continuación viene ejemplos de pasos:
	% En el siguiente paso: "Ingresa al sistema" es el objetivo del paso y "escribiendo la URL de la aplicación" es la acción en concreto.
	\UCpaso[\UCactor] Ingresa al sistema desde la \IUref{IU003}{Pagina Principal}.
	% En el siguinte paso se referencia una Interfaz
  \UCpaso Obtiene todos los estados activos y las formaciones guardadas para las listas desplegables
	\UCpaso \label{UC4.Datos} Solicita al usuario las entradas en la  \IUref{IU005}{Registrar Centro de Trabajo}
	% En el siguiente paso está etiquetado para ser referenciado por un error o trayectoria alternativa:
	\UCpaso[\UCactor]  Introduce \hyperlink{CT.Calle}{Calle}, \hyperlink{CT.Numero}{Numero} , \hyperlink{CT.Colonia}{Colonia} , \hyperlink{CT.Municipio}{Municipio} 
    \UCpaso[\UCactor] Selecciona de las listas desplegables \hyperlink{Estado.Nombre}{Estado} , \hyperlink{Formacion.Nombre}{Formación} 
	% En el siguiente paso se usa el comando \IUbutton 
	\UCpaso[\UCactor] Solicita el registro en el sistema presionando el botón \IUbutton{Registrar}.
	% En el siguiente paso se referencía un error.
	 \UCpaso Verifica que todos los campos obligatorios esten llenos \ErrorRef{CU004}{E1}{Campos faltantes}.
  \UCpaso Obtiene el ID del Estado seleccionado.
  \UCpaso Obtiene el ID de la formacion seleccionada.
  \UCpaso Le asigna a este Centro de Trabajo el número siguiente disponible como su idetificador.
  \UCpaso Verifica el formato en el campo CCT \ErrorRef{CU004}{E2}{Campo con formato incorrecto}.
  \UCpaso Verifica si la CCT no esta registrada en el sistema \Trayref{CU004}{A}.
  
   \UCpaso Guarda la información del CT.
  \UCpaso Guarda la información del CCT.
 
	% En el siguiente paso se señala una trayectoria alternativa
	\UCpaso \label{UC4.Mensaje} Muestra la pantalla \IUref{IU5}{Registrar Centro de Trabajo} con el mensaje \MSGref{MSG-011}{Registr exitoso} y los campos vacios.
\end{UCtrayectoria}


%--------------------------------------
% Las trayectorias alternativas se identifican con Letras: A, B, C, etc.
\begin{UCtrayectoriaA}{CU004}{A}{La CCT ya esta registrada en el sistema}
	\UCpaso Verifica si la CCT puede ser prestada. \ErrorRef{CU004}{E3}{Clave de Centro de Trabajo no prestada}
    \UCpaso Guarda la información del CT.
    \UCpaso Obtiene el identificador del centro de trabajo.
    \UCpaso Obtiene la fecha actual.
    \UCpaso Obtiene la fecha de termino del ciclo escolar más proximo.
    \UCpaso Guarda la información del prestamo. 
    % Se puede desprender otra trayectria alternativa si es necesario.
	% Finalice la trayectoria indicando si la ejecución se integra a la trayectoria anterior o si termina la ejecución del CU.
	% Verifique que la redacción de la trayectoria deje en claro si el objetivo del CU se alcanzó o no.
	\UCpaso[] El Caso de Uso continúa en el paso \ref{UC4.Mensaje}.
\end{UCtrayectoriaA}


%--------------------------------------
% Puntos de extensión

% Comente la siguiente sección en caso de que no hayan puntos de extensión o relaciones de tipo extends.

		
		
		
%-------------------------------------- TERMINA descripción del caso de uso.