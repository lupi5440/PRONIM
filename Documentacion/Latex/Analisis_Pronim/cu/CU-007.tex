%!TEX root = ../proyecto.tex

% Plantilla para caso de uso sencillo con ejemplos de comandos e intrucciones.
%-------------------------------------- COMIENZA descripción del caso de uso.

%\begin{UseCase}[archivo de imágen]{UCX}{Nombre del Caso de uso}{
%--------------------------------------
\begin{UseCase}{CU-007}{Asignar Director}{
	% La descripción debe describir el evento de inicio del caso de uso, Breve descripción de la trayectoria y estado final del caso de uso.
}
	\UCitem{Versión}{\color{Gray}
		0.1	% Ponga un número de versión, 
	}\UCitem{Autor}{\color{Gray}
		Agustin Flores Lopez. % Analista responsable de especificar el CU
	}\UCitem{Supervisa}{\color{Gray}
		Nombre del analista revisor. % Analista responsable de verificar que está correcto.
	% TODO: Dar de alta al actor Usuario
	}\UCitem{Actor}{
		\hyperlink{CoordinadorEstatal}{Coordinador Estatal} % No olvide dar de alta el actor.
	}\UCitem{Propósito}{\begin{Titemize}%Indique los fines, objetivos, propósitos o valores agregados del Caso de uso.
		\Titem Permitir a un director realizar sus facultades.
		\Titem Actualizar quien dirige un Centro de Trabajo.
	\end{Titemize}
	}\UCitem{Entradas}{\begin{Titemize}
		% TODO: Dar de alta las entidades que se listan.
		\Titem \hyperlink{CT.Nombre}{Nombre} del Centro de Trabajo. % El identificador no acepta acentos, espacios ni eñes.
		\Titem \hyperlink{Profesor.ID}{Identificador} de profesor.
        \Titem \hyperlink{Profesor.FechaFin}{Fecha de termino} del periodo de director.% Liste todos los datos de entrada
		\end{Titemize}
	}\UCitem{Origen}{\begin{Titemize}
		\Titem Se introducen desde el teclado. % Indique por que medio se introducen los datos, 
		\Titem otros.          % Si es ḿas de uno indique que datos corresponden en cada medio de entrada.
	\end{Titemize}
	}\UCitem{Salidas}{\begin{Titemize}
		% TODO: Dar de alta las entidades que se listan.
		\Titem Mensajes de error. % Indique por que medio se introducen los datos, 
	\end{Titemize}
	}\UCitem{Destino}{\begin{Titemize}
		\Titem Se muestra en la pantalla \IUref{IU007}{Asignar Director }.. % Indique por que medio se muestran los datos, 
		          % Si es ḿas de uno indique que datos corresponden en cada medio de entrada.
	\end{Titemize}
	}\UCitem{Precondiciones}{\begin{Titemize}
		% Incluya Precondiciones lógicas, de negocio e incluso las que debe atender el usuario. 
		% Muchas precondiciones provienen de reglas de negocios, otras estarán asociadas a manejo de errores
		% Otras están relacionadas con casos de uso que deben ejecutarse previamente, como registrar un producto.
		\Titem El usuario debe haber iniciado sesión.
  \Titem Debe haber al menos un profesor registrado.
  \Titem Debe haber al menos un centro de trabajo registrado.
	\end{Titemize}
	}\UCitem{Postcondiciones}{\begin{Titemize}
		% Indique todas las postcondiciones
		% por ejemplo, Cambios en el sistema
		% Cambios en la BD una vez terminado el CU
		% Efectos colaterales
		% Condiciones de término.
		\Titem Se guardara la fecha de inicio y termino de un director asi como su centro de trabajo.
	\end{Titemize}
	}\UCitem{Errores}{\begin{Titemize}
		% Escriba todos los errores que puedan ocurrir en el sistema, para cada error recuerde:
		% Punerle un identificador
		% Describir la condición o escenario que detona el error
		% Describa la forma en que debe reaccionar el sitema: si la reaccion corresponde a varios pasos use mejor una trayectoria alternativa.
		% Relacione el error con la trayectoria principal.
		\Titem {\bf \hypertarget{CU005.E1}{E1}}: La fecha proporcionada es anterior a la actual, el sistema muestra el \MSGref{MSG-0012}{Fecha novalida} y regresa al paso \ref{UC5.Datos}.
		\Titem {\bf \hypertarget{CUX.E2}{E2}}: Condición que detona el error, reacción del sistema y termina el Caso de uso..
	\end{Titemize}
	}\UCitem{Tipo}{
		% Especifique el tipo de caso de us, puede ser: "Caso de uso primario" o 
		% "Viene de \\hyperref{CUY}{CUY nombre del CU}" cuando se desprende desde otro caso de uso mediante un extends.
		Caso de uso primario
	}\UCitem{Observaciones}{
		% Indique las observaciones al caso de uso, las cuales pueden ser:
		% - Ninguna
		% - Dudas sobre el procedimiento o la especificación.
		% - Issues detectados
		% - Suposiciones realizadas.
		% - Cualquier otra especificacion que considere pertinente que no pudo colocarse en los demás atributos del Caso de uso
		% - Aclaraciones.
		% - Notas para el usuario o desarrollador.
		% - Pendientes (TODO's) en caso de no usar los comentarios.
	}
\end{UseCase}

%--------------------------------------
\begin{UCtrayectoria}
	% Cada paso debe inicair con un Verbo en infinitivo, siempre especificando el objetivo del paso mas la accion en concreto.
	% \UCpaso[\UCactor] se refiere al actor y \UCpaso se refiere al sistema.
	% A continuación viene ejemplos de pasos:
	% En el siguiente paso: "Ingresa al sistema" es el objetivo del paso y "escribiendo la URL de la aplicación" es la acción en concreto.
	\UCpaso[\UCactor] Ingresa al sistema desde la \IUref{IU003}{Pagina Principal}.
	% En el siguinte paso se referencia una Interfaz
  \UCpaso Obtiene todos los centros de trabajo del estado del actor y los profesores activos.
	\UCpaso \label{UC5.Datos} Solicita al usuario las entradas en la  \IUref{IU007}{Registrar Centro de Trabajo}
	% En el siguiente paso se usa el comando \IUbutton 
	\UCpaso[\UCactor] Selecciona los datos desde los desplegables.
    \UCpaso[\UCactor] Presiona el \IUbutton{Aceptar}
    \UCpaso obtiene el id del centro de trabajo seleccionado.
    \UCpaso verifica que la fecha proporcionada sea posterior a la del dia de hoy \ErrorRef{CU005}{E1}{Fecha anterior a la actual}.
    \UCpaso Guarda el nuevo director.
	% En el siguiente paso se referencía un error.
	\
\end{UCtrayectoria}


%--------------------------------------
% Las trayectorias alternativas se identifican con Letras: A, B, C, etc.



%--------------------------------------
% Puntos de extensión

% Comente la siguiente sección en caso de que no hayan puntos de extensión o relaciones de tipo extends.

		
		
		
%-------------------------------------- TERMINA descripción del caso de uso.