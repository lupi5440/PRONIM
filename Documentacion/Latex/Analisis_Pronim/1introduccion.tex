% !TeX root = proyecto.tex

%=========================================================
\chapter{Introducción}


% \cdtInstrucciones{Presentar el documento, indicando su contenido, a quien va dirigido, quien lo realizó, por que razón, dónde y cuando. }

Este documento contiene el análisis de requerimientos del proyecto ``{\em Nombre del proyecto}'' que servirá como base para el análisis, diseño, construcción, pruebas y aceptación del proyecto.

%---------------------------------------------------------
\section{Presentación}

% \cdtInstrucciones{Presente en un par de párrafos el contexto y el problema en que se define el proyecto.}

El Programa de Educación Básica para Niños y Niñas de Familias Jornaleras Agrícolas Migrantes (PRONIM) brinda atención educativa a niñas y niños de familias jornaleras agrícolas migrantes y/o asentadas, de 3 a 16 años de edad. Opera en los centros educativos ubicados en las comunidades y en los campamentos agrícolas de destino de esta población, en ellos brindan las condiciones para que con la participación de docentes, asesores escolares, asesores técnico-pedagógicos, se lleve a cabo una atención educativa de calidad.
Sin embargo la movilidad entre los distintos Centros de Trabajo que estan acobijados dentro del PRONIM es complicada de realizar debido a la falta de comunicación entre los distintos Centros de Trabajo, este pequeño problema genera grandes afectaciones al desarrollo de los niños y niñas registradas en el programa.

% \cdtInstrucciones{Indique el propósito del documento, a quien va dirigido y como debe ser utilizado el documento.}

El propósito del presente documento es mantener la comunicación entre los miembros del equipo de desarrollo además de generar los acuerdos necesarios para la aceptación del mismo, este documento debe ser actualizado para mantener su correcto funcionamiento.

%---------------------------------------------------------
\section{Organización del contenido}

% \cdtInstrucciones{Indique el contenido y organización del documento.}

En el capítulo \ref{cap:alcance} podemos encontrar la información relativa al alcance propuesto para este proyecto que incluye el analisis de la problematica, problemas identificados,contexto, analisis de las causas y consecuencias, caracteristicas y sintesis de la solución, asi como los objetivos del proyecto, los usuarios del sistema, procesos que se veran afectados por la adopción del sistema, requerimientos del usuario y la especificación de la plataforma.

En el capítulo \ref{cap:reqSist} Se encontrara el model del negocio, haciendo incapie a los actores del sistema, los terminos de negocio, las reglas de negocio y las maquina de estados.

En el capítulo \ref{cap:modDinamico} Se encontrara el model dinamico, con los casos de uso y trayectorias de los mismos.

En el capítulo \ref{cap:modInteraccion} Se encontrara los modelos de navegacion, las interfaces de usuario junto sus caracteristicas y los mensajes \ref{sec:mensajes} que se enviaran entre los distintos componentes del sistema.

%---------------------------------------------------------
\section{Notación, símbolos y convenciones utilizadas}

\cdtInstrucciones{
	Indique la notacion utilizada así como nórmas o estándares de documentación utilizados en el documento.
}
Todos los diagramas incluidos en este documento que sean posible modelar bajo el marco de UML seran modelados con base en sus normas.
Los terminos especificos para el negocio seran descritos en \ref{sec:terminosDeNegocio}
